\documentclass{article}

\usepackage{amsmath}

\begin{document}
\begin{flushleft}
$P_0=$ Initial Loan Size (\$)

$R=$ Annual interest rate (APR) described as an annual number (\% 0-1)

$r=R/12$ Interest rate per month (\% 0-1)

$M=$ Monthly Payment (\$)

$n=$ Number of payments (months) \\
\vspace{5mm}
Each month there is an interest payment on the principal. Anything extra beyond the interest payment will deduct from the principal. For month 1:
\begin{equation}
\textrm{Interest owed} = P_0 * r
\end{equation}
Since the principal is reduced by payment beyond the interest, the principal is reduced by 
\begin{equation}
M-P_0 r
\end{equation}
So the principal after the first payment will be 
\begin{equation}
\begin{aligned}
P_1 &= P_0 - (M - P_0 r) \\
& = P_0 (1+r) - M
\end{aligned}
\end{equation}
And the principal after the second payment will be
\begin{equation}
\begin{aligned}
P_2 &= P_1 - (M - P_1 r) \\
&= P_1 (1+r) - M
\end{aligned}
\end{equation}
And the third payment will be
\begin{equation}
\begin{aligned}
P_3 &= P_2 - (M - P_2 r) \\
&= P_2 (1+r) - M
\end{aligned}
\end{equation}
Clearly, there is a pattern, the following recursive formula is true:
\begin{equation}
P_{n + 1} = P_n (1 + r) - M
\end{equation}
Right now, this is not useful. If we substitute $(4)$ into $(5)$ we get
\begin{equation}
\begin{aligned}
P_3 & = (P_1 (1 + r) - M)(1 + r) - M \\
& = P_1 (1 + r)^2 - M(1 + r) - M
\end{aligned}
\end{equation}
And substituting $(3)$ into $(7)$ gives
\begin{equation}
\begin{aligned}
P_3 & = (P_0 (1 + r) - M)(1 + r)^2 - M(1 + r) - M \\
& = P_0 (1 + r)^3 - M(1 + r)^2 - M(1 + r) - M
\end{aligned}
\end{equation}
The pattern here is 
\begin{equation}
P_n = P_0 (1 + r)^n - [M + M(1 + r) + M( 1 + r)^2 + ... + M(1 + r)^{n - 1}]
\end{equation}
The M terms are going through a geometric series. The factor by which M is multiplied is $(1 + r)$. For a standard geometric series:
\begin{equation}
\sum_{n = 0}^{\infty} ar^n = a \left( \frac{1 - r^n}{1 - r} \right)
\end{equation} 
and substituting values from $(9)$ gives
\begin{equation}
\begin{aligned}
P_n & = P_0 (1 + r)^n - \sum_{n=0}^{\infty} M(1 + r)^n \\
& = P_0 (1 + r)^n - \left[ M \left( \frac{1 - (1 + r)^n}{1 - (1 + r)} \right) \right]
\end{aligned}
\end{equation}
We are interested in when $P_n = 0$ because this is when the amount owed is 0. Now we just need to solve for $M$ when $P_n = 0$. Rearranging $(11)$,
\begin{equation}
M \left( \frac{(1 + r)^n - 1}{r}\right) = P_0 (1 + r)^n
\end{equation}
so
\begin{equation}
\boxed {
M = \frac{r P_0 (1 + r)^n}{(1 + r)^n - 1}
}
\end{equation}
\end{flushleft}
\end{document}




















